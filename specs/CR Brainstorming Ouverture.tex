\documentclass[a4paper, 12pt]{report}

\usepackage{lmodern} % Police standard sous LaTeX : Latin Modern
% (alternative à la police d'origine développée par Donald
	%Knuth : Computer Modern)
\usepackage[french]{babel} % Pour la langue fran¸caise
\usepackage[utf8]{inputenc} % Pour l'UTF-8
\usepackage[T1]{fontenc} % Pour les césures des caractères accentués
\renewcommand{\thesection}{\Roman{section}}
\usepackage{amssymb}
\usepackage{enumitem}
\usepackage{pifont}
\usepackage{xcolor}
\usepackage{stmaryrd}
\usepackage{amsmath,amsfonts, amssymb}
\usepackage{graphicx}

\begin{document}

\section*{Compte-rendu réunion 1 : Brainstorming d'ouverture}

\textbf{Présents :} tout le monde (Dupret ; Georgeon ; Couturier ; Mandrelli)
\\

\textbf{Objectifs de la séance :} définir les grandes lignes de notre projet (référence à un thème que l'on veut aborder et éventuellement un jeu déjà existant en référence) ; scénario ou pas ; type de jeu (FPS, RPG) ; date de prochaine réunion.\\

\textbf{Idées proposées}
\begin{itemize}

\item Dupret : jeu de plateforme en 2D (Mario) \& RPG (Pokémon) \& \textit{Visual Novel}\\
Souci : pour le jeu de plateforme, il faut des idées innovantes pour éviter de rendre le jeu "peu inspiré"
\\

\item Georgeon : jeu de combat en vue du dessus \textit{rogue-lite arena shooter} (Brotato)\\
Techniques d'IA pour les ennemis envisageables \& potentiel de jeu infini (scénario ou pas, à voir)\\

\item Couturier : Jeu éducatif d'une certaine manière (apprentissage de l'anglais vàv du 2nd projet en 2A, ou codage en C) => RPG vue du dessus (Pokémon) ; FPS éventuellement (prendre Télécom comme environnement) + style \textit{backroom}\\
Pédagogie pour apporter une plus-value au jeu et partir des consignes\\
Map linéaire dans un faux TN où on a un scénario basique. Il faut avancer entre des vagues de mobs (cf idées \textit{rogue-lite}) et des "salles de repos" dans lesquelles on retrouve l'aspect Visual Novel => intégration d'un aspect scénario \& 
\\

\item Mandrelli : RPG \& plateformeur 3D (Honkai Star Rail) => éventuellement faire un \textit{de-pake} du mode "Pure Fiction"\\
Vagues d'ennemis infinies en vue 2D (vue classique de RPG) ; plus des vagues sont vaincues, plus des objets d'améliorations sont disponibles mais pour les débloquer, résoudre des "énigmes" (cf jeu éducatif) ou afficher l'équation de calcul des boosts et laisser le joueur expérimenter les différentes combinaisons.\\

\end{itemize}

\textbf{Idée favorisée pour l'instant :}
\begin{itemize}[label=$\bullet$]

\item Zone sécurisée et "relaxante" qui fait office de Hub dans lequel on retrouve l'aspect \textit{Visual Novel} (\& système éducatif éventuellement).\\
Map $\pm$ linéaire (déblocage de la carte linéaire) ; et les zones déjà finies ne servent (a priori) plus à l'intrigue. Les zones à explorer sont remplies d'ennemis qui seront affrontés au Tour-par-Tour.
\\

Histoire / Scénario d'environ une heure. La fin du jeu se déroulera durant un combat (en TpT) dans le Hub contre le boss final.
\\

\item \textbf{Gameplay concret :} trois axes de jeu : tour-par-tour pour les combats ; explorations en temps réel pour le reste des missions ; \textit{Visual Novel} pour les entre-missions.
\\

\item Style graphique : \textit{pixel-art}
\\

\item \textit{Idées supplémentaires : système de vagues infinies en défi supplémentaire à la fin du jeu (pour le tryhard), et re-déblocage des zones déjà faites mais en mode plus difficile.\\}

\end{itemize}

\underline{Date de la prochaine réunion :\\}
\begin{center}
\textbf{Mercredi 21 Février - 16h - Salle de Travail ou Médialab}\\

\textbf{À faire :} définir la date d'envoi du mail de lancement ; définition des grandes lignes du scénario, de l'univers de jeu et éventuellement de quelques personnages ; définir les rôles de chacun
\end{center}


\end{document}